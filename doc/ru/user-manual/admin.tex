
\section{Администрирование \ds}

Для~правильной работы \ds необходимо уделять достаточное внимание задачам его администрирования.
Они должны включать как задачи обслуживания \ds на~рабочих местах, так и задачи обслуживания репозиториев пакетов,
которые служат источниками программ для доставки и установки на~компьютеры пользователей.

\subsection{Администрирование репозиториев пакетов}

Пакеты для~установки размещаются на~узлах в~сети и доставляются по~одному из~распространённых протоколов.
Сами~по~себе файлы пакетов перед публикацией никакой дополнительной обработки проходить не~должны,
требуется только правильное именование каталогов, в~которых они размещаются (см.~разд.~\ref{repo_format}).
Все~задачи администрирования репозиториев связаны с~созданием и поддержкой в~актуальном состоянии специальной вспомогательной информации,
здесь и далее называемой ``индексом  репозитория''.
Индекс используется для~хранения подробной информации о~наборе пакетов в~репозитории и их~содержимом
Подробное описание формата индекса можно найти в~разделе \ref{repo_format} спецификаций,
здесь мы рассмотрим порядок работы основных утилит для~администрирования репозиториев \ds.

