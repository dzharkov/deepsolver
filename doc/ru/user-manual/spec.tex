
\section{Спецификации}

\subsection{Репозитории пакетов}
\label{repo_format}

Репозиторий пакетов \ds может содержать два~уровня разбиения.
Назначение каждого из~них администратор должен определить самостоятельно.
Если в~качестве одного из~них выбрано разделение по~архитектуре процессора, то рекомендуется использовать для~этого первый уровень.

На~первом уровне разделение ведётся по~имени каталога. 
Каждый каталог должен иметь имя, содержащее только латинские буквы, цифры, дефис и знак подчёркивания.
На~втором уровне разбиение производится путём создания нескольких каталогов,
имена которых присвоены по~следующим правилам:

\begin{enumerate}  

\item {
\CODE{base.имя} --- каталог для~индексов компоненты репозитория;
}

\item {
\CODE{RPMS.имя} --- каталог для~бинарных пакетов;
}

\item {
\CODE{SRPMS.имя} --- каталог для~пакетов с~исходными текстами.
}

\end{enumerate}

Например, если требуется создать репозиторий для~архитектур \EN{noarch} и \EN{i586}, 
для~двух компонент \EN{main} и \EN{debuginfo},
то перечень каталогов должен быть следующим:

\begin{itemize}

\item{\CODE{i586/base.main};}
\item{\CODE{i586/base.debuginfo};}
\item{\CODE{i586/RPMS.main};}
\item{\CODE{i586/RPMS.debuginfo};}
\item{\CODE{i586/SRPMS.main};}
\item{\CODE{i586/SRPMS.debuginfo};}
\item{\CODE{noarch/base.main};}
\item{\CODE{noarch/base.debuginfo};}
\item{\CODE{noarch/RPMS.main};}
\item{\CODE{noarch/RPMS.debuginfo};}
\item{\CODE{noarch/SRPMS.main};}
\item{\CODE{noarch/SRPMS.debuginfo}.}

\end{itemize}
