
\section{Спецификации}

\subsection{Репозитории пакетов}
\label{repo_format}

Репозиторий пакетов \ds может содержать два~уровня разбиения.
Назначение каждого из~них администратор должен определить самостоятельно.
Если в~качестве одного из~них выбрано разделение по~архитектуре процессора, то рекомендуется использовать для~этого первый уровень.

На~первом уровне разделение ведётся по~имени каталога. 
Каждый каталог должен иметь имя, содержащее только латинские буквы, цифры, дефис и знак подчёркивания.
На~втором уровне разбиение производится путём создания нескольких каталогов,
имена которых присвоены по~следующим правилам:

\begin{enumerate}  

\item {
\PATH{base.имя} --- каталог для~индексов компоненты репозитория;
}

\item {
\PATH{RPMS.имя} --- каталог для~бинарных пакетов;
}

\item {
\PATH{SRPMS.имя} --- каталог для~пакетов с~исходными текстами.
}

\end{enumerate}

Например, если требуется создать репозиторий для~архитектур \EN{noarch} и \EN{i586}, 
для~двух компонент \EN{main} и \EN{debuginfo},
то перечень каталогов должен быть следующим:

\begin{itemize}

\item{\PATH{i586/base.main};}
\item{\PATH{i586/base.debuginfo};}
\item{\PATH{i586/RPMS.main};}
\item{\PATH{i586/RPMS.debuginfo};}
\item{\PATH{i586/SRPMS.main};}
\item{\PATH{i586/SRPMS.debuginfo};}
\item{\PATH{noarch/base.main};}
\item{\PATH{noarch/base.debuginfo};}
\item{\PATH{noarch/RPMS.main};}
\item{\PATH{noarch/RPMS.debuginfo};}
\item{\PATH{noarch/SRPMS.main};}
\item{\PATH{noarch/SRPMS.debuginfo}.}

\end{itemize}

Каталог \PATH{base.*}, предназначенный для~хранения индекса, должен содержать следующие файлы:

\begin{itemize}

\item{\PATH{info} --- информационный файл с~параметрами индекса;}
\item{\PATH{.rpms.complete.data} --- вспомогательный файл, не~предназначенный для~загрузки пользователями, с~информацией для~повторной фильтрации \provides;}
\item{\PATH{rpms.data} --- основной список пакетов с~информацией о~зависимостях между ними;}
\item{\PATH{rpms.descr.data} --- список пакетов с~расширенными описаниями;}
\item{\PATH{rpms.filelist.data} --- списки файлов для~каждого бинарного пакета;}
\item{\PATH{srpms.data} --- основная информация о~пакетах с~исходными текстами;}
\item{\PATH{srpms.descr.data} --- список пакетов с~исходными текстами, содержащий расширенную информацию.}

\end{itemize}
