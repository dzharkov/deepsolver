
\section{Общие требования}

Система для~управления \EN{rpm}-пакетами должна состоять из~нескольких отдельных утилит,
набор которых подробно описывается в~последующих главах.
При~этом существует ряд общих требований, предъявляемых к~любому приложению из~всего множества.
Ниже приводится их~подробный перечень.

\begin{description}

\item[Поддержка протоколирования.]
Любая операция, приводящая к~изменению состояния системы, 
должна выполнять подробное протоколирование всех совершаемых  действий.
Создаваемый журнал необходимо дополнять как машинночитаемой информацией,
с~целью обеспечения автоматизированной возможности возврата ОС к~предыдущему состоянию в~случае неудачного завершения транзакции, 
так и информацией, облегчающей чтение и поиск данных системным администратором.
Записи протоколирования должны разделяться по~степени критичности содержания 
(уровни ``отладка'', ``основная информация'', ``ошибка'', ``критическая ошибка'').
Для~всех утилит требуется наличие ключа \CODE{--verbose} повышающей детальность информации,
выводимой на~стандартный поток вывода.

\item[Исключение возможности злоумышленного подмена данных.]
Конечный пользователь всей системы должен иметь возможность удостовериться, что множество пакетов для~установки в~ОС и 
вспомогательная информация о~них предоставлены доверенным лицом.
Контроль аутентичности на~основе цифровой подписи автора необходимо выполнять автоматически  во~время работы.
При~обнаружении признаков атаки злоумышленника требуется уведомить об~этом пользователя и добавить соответствующую информацию в~журнал.
Уведомление должно указывать, обнаружено~ли только несовпадение контрольной суммы (возможное случайное повреждение, например, при~передаче по~сети)
или найдена недействительная цифровая подпись при~корректной контрольной сумме (достоверно злоумышленный характер подмены).

\item[Минимизация риска аварийного завершения операции.]
При~любой операции, предполагающей внесенеи изменений в~состояние ОС, необходимо предварительно 
провести максимально доступное количество проверок с~целью предотвращения неуспешного результата работы.
Начало модификации системы допускается только в~случае, если все~необходимые данные доставлены на~локальный компьютер,
проверена их~целостность, устранена возможность файловых конфликтов (см.~гл.~\ref{pkgstruct}), и доступно достаточное количество свободного дискового пространства.

\item[Поддержка параллельных вычислений.]
Все~операции, предполагающие высокую вычислительную нагрузку на~процессор,
по~мере возможности должны поддерживать параллельный режим работы.
Разделение общей задачи на~несколько процессов или потоков позволить задействовать несколько ядер в~многоядерных системах,
что способно значительно ускорить получение результата. 
Желательно, чтобы количество процессов или потоков в~вычислении не~превышало~бы физическое количество ядер системы.
Это требование не~распространяется на~процедуру загрузки данных по~сети.

\item[Кэширование данных из~удалённых источников]
В~случае загрузки пакетов для~установки из~удалённых репозиториев в~сети
полученные файлы должны сохраняться в~специально отведённой системной директории на~случай повторного использования.
При~этом требуется наличие конфигурационного параметра, задающего максимальной размер директории для~хранения.
В~случае превышения заданного размера пакеты необходимо удалять с~учётом давности последнего обращения.

\end{description}

\section{Основные операции}
\subsection{Операция установки и обновления пакетов}

Операции установки и обновления пакетов имеют единый интерфейс пользователя.
В~качестве основного параметра указывается множество пакетов, каждый из~которых может быть  задан одним из~следующих способов:

\begin{itemize}

\item {по~имени;}
\item {по~имени с~указанием версии и релиза, включая возможность даунгрейда;}
\item {по~ссылке на~файл пакета, включая возможность загрузки из~сети при~помощи \EN{URL};}
\item{по~одному из \EN{provides} пакета (см.~гл.~\ref{pkgstruct}), включая возможность указания версии.}

\end{itemize}

При~указании более одного пакета операция отменяется полностью для~\underline{всех} них, 
если обнаружена проблема установки хотя~бы одного.
При~вызове операции необходимо наличие возможности запрета рассмотрения решений, 
включающих установку некоторого пакета. 

Предполагается, что два~пакета с~одинаковыми именами, эпохами, версиями и релизами являются одним и~тем~же пакетом,
и не~имеет значения, какой из~них будет выбран для~установки.
Только в~случае использования точной версии или имени файла происходит однозначный выбор пакета для~установки.
В~случае отсутствия версии или указания ограничения версии (не~новее заданной или не~старше заданной)
пакет для~установки должен определяться следующим способом:

\begin{enumerate}

\item{
Если существует пакет, имя которого соответствует заданному, ему отдаётся приоритет.
Существующие пакеты, содержащие одноимённые \EN{provides}, не~рассматриваются.
}

\item {
Если указанному имени соответствуют только \EN{provides} других пакетов,
то приоритет определяется на~основе предустановленных параметров (см.~гл.~\ref{advfeatures}).
Прочие альтернативы используются только в~случае невозможности установки приоритетного варианта (без~запроса пользователю).
При~отсутствии указания приоритета порядок выбора не~определён.
}

\item {
При~наличии нескольких версий пакета приоритет отдаётся самой старшей.
В~случае невозможности её~установки производится выбор наиболее свежей версии, приводящей к~возможному решению, с~запросом подтверждения пользователю.
При~обновлении пакета могут рассматриваться только версии старше имеющейся.
Если пакет был отобран на~основе его~\EN{provides}, 
то версия не~анализируется\footnote{Возможность использования версии самого пакета не~рассматривается,
поскольку один и~тот~же \EN{provides} могут иметь не~связанные между собой пакеты, сравнение версий которых бессмысленно.
Версии \EN{provides} сравнивать невозможно, поскольку они могут быть не~только конкретным значением, но и множеством значений.}
}

\end{enumerate}

При~рассмотрении множества пакетов для~разрешения \EN{provides}
анализируются \EN{provides} только с~версией, если пользователь наложил ограничения версии.
В~противном случае требуется рассмотрение всех \EN{provides} с~соответствующем именем.

В~случае обнаружения невозможности выбора самой старшей версии хотя~бы для~одного пакета из~указанного множества
должен производиться поиск первого допустимого решения с~перебором всех доступных версий всех указанных при~вызове пакетов,
затем пользователю необходимо предложить подтверждение установки всех пакетов, для~которых выбрана не~самая старшая версия из~доступных.
Если пользователь отвечает отрицательно на~хотя~бы один из~вопросов, работа приложения останавливается без рассмотрения прочих вариантов.

При~указании множества пакетов для~установки или обновления  должна допускаться возможность использования регулярных выражений.
С~этой целью требуется наличие аргумента командной строки,  разрешающего обработку ввода пользователя как~регулярных выражений,
а~также определяющего,  должны~ли регулярные выражения обрабатываться на~множестве имён пакетов или также с~учётом известного списка всех \EN{provides}.
Все~обнаруженные допустимые подстановки должны передаваться на~вход операции так, как если~бы пользователь сделал перечисление вручную.

Необходимо наличие возможности следующих вариантов выполнения операции без~изменения состояния ОС:

\begin{enumerate}

\item {
Произвести поиск решения и вывести перечень требуемых изменений на~стандартный поток вывода,
включая возможность описания в~машинночитаемом формате.
При~указании списка пакетов для~загрузки из~сети по~желанию пользователя необходимо указывать полные \EN{URL} к~файлам.
}

\item {
Произвести поиск решения и выполнить только доставку необходимых пакетов из~сети в~кэш или в~директорию, указанную пользователем.
}

\end{enumerate}

В~стандартном поведении система должна производить поиск решений с~учётом экономии дискового пространства пользователя. 
Дополнительно требуется наличие параметра командной строки, запрещающего удаление уже~установленных пакетов из~системы. 

\subsection{Операция удаления пакетов}

При~выполнении операции удаления указывается множество пакетов для~обработки.
Допускается использование только настоящих имён пакетов, \EN{provides} не~рассматриваются.
Операция производит изменения, после которых ни~один из~указанных пакетов не~может быть установленным в~ОС,
но допускается установка дополнительных.
Ситуация вызова операции для~отсутствующего пакета ошибочной не~считается.

Для~увеличения точности определения желаемого результата требуется возможность указания, 
что некоторый пакет должен присутствовать в~системе после окончания работы.
Необходимо наличие опции командной строки, запрещающей установку новых пакетов. 

Должны быть предусмотрены следующие варианты работы без~изменения состояния ОС:

\begin{enumerate}

\item {
Произвести поиск решения и вывести перечень требуемых изменений на~стандартный поток вывода,
включая возможность описания в~машинночитаемом формате.
При~указании списка пакетов для~загрузки из~сети по~желанию пользователя необходимо указывать полные \EN{URL} к~файлам.
}

\item {
Произвести поиск решения и выполнить только доставку необходимых пакетов из~сети в~кэш или в~директорию, указанную пользователем.
}

\end{enumerate}

\subsection{Операция переустановки пакетов}

Операция переустановки одного или нескольких пакетов производится для~исправления повреждённой ОС.
В~ходе работы пакет удаляется и заново устанавливается, сохраняя версию.
При~отсутствии необходимого пакета в~кэше производится  его~загрузка из~доступных репозиториев.
Требуется наличии режима работы только для~доставки необходимого пакета в~кэш или в~директорию, указанную пользователем.

\subsection{Операция обновления ОС}

Операция обновления ОС соответствует операции обновления пакетов с~указанием множества имён всех~установленных пакетов .
Сохраняется возможность пробного запуска работы только с~выводом списка изменений на~поток стандартного вывода или с~загрузкой пакетов из~репозиториев в~сети.
В~случае обнаружения невозможности выполнения операции пользователю необходимо предоставить подробный отчёт с~описанием проблемы и предложить сделать дополнительные указания для~её~разрешения.

\subsection{Операция получения исходных текстов}

При~вызове операции получения исходных текстов пользователь должен указать имя одного пакета,
для~которого он~желает получить исходные тексты.
В~ходе работы требуется произвести в~хранимой базе данных   поиск имени пакета с~исходными текстами,
из~которого был получен указанный бинарный пакет, 
определить его~местонахождение в~имеющихся репозиториях и доставить в~директорию, указанную пользователем.
Загруженный пакет в~кэш не~помещается.

\subsection{Операция исправления целостности ОС}

Операция контроля целостности ОС должна провести для~каждого пакета  проверку удовлетворения его~зависимостей и конфликтов с~другими пакетами.
Процедуру исправления найденных ошибок требуется выполнять следующим образом:

\begin{enumerate}

\item {
Для~каждой пары конфликтующих пакетов пользователю необходимо предоставить запрос для~выбора, 
какой из~пакетов должен быть удалён из~ОС.
}

\item {
Дополнить список пакетов для~удаления списком пакетов для~установки,
добавляя в~него все неудовлетворённые зависимости.
}

\item {
При~поиске исправлений должны одновременно учитываться и пакеты для~установки, и пакеты для~удаления.
В~случае невозможности нахождения решения пользователю необходимо предоставить подробное описание с~предложением установить или удалить часть пакетов вручную,
после чего повторить операцию исправления целостности.
}

\end{enumerate}

Необходимо наличие возможности выполнения операции в~информационном режиме с~выводом на~поток  стандартного вывода списка необходимых изменений,
а~также в~режиме доставки недостающих пакетов в~кэш или в~указанную директорию.

\subsection{Дополнительные возможности}
\label{advfeatures}

Требуется наличие ряда возможностей, модифицирующих поведение операций, изменяющих состояние ОС.
Они подразумевают указание пользователем дополнительной информации,
которая должна сохраняться в~конфигурационных файлах. 

\begin{description}

\item[Указание приоритетов разрешения \EN{provides}.]
Пользователь должен иметь возможность составить список приоритетного выбора пакетов при~разрешении \EN{provides}.
Список просматривается от~начало, и переход к~следующему элементу происходит только в~случае явной невозможности найти решение с~использованием предыдущих значений.
За~пределами списка, если ни~одно из~указанных значений не~приводит к~допустимому решению,
порядок разрешения \EN{provides} не~определён.

\item[Указание списка пакетов для~удержания.]
Пользователь должен иметь возможность указать список пакетов, для~которых запрещается операция обновления.
Ни~одна из~операций, изменяющих состояние ОС, ни~может удалить или обновить пакет, указанный в~списке для~удержания,
за~исключением случаев явного перечисления пакета пользователем при~вызове операции.

\item[Указание списка ``важных'' пакетов.]
Пользователь должен иметь возможность определить список пакетов, 
удаление которых приводит ОС в~неработоспособное состояние.
Пакеты из~указанного списка не~могут удаляться в~ходе поиска решений, а в~случае явного указания их~пользователем в~операции удаления
необходимо запросить подтверждение, что пользователь понимает риск и последствия удаления запрошенного пакета.

\end{description}

\section{Информационные операции}

Наряду с~операциями, предполагающими внесение изменений в~ОС,
необходимо наличие нескольких информационных операций,
предоставляющих различные данные по~запросу пользователя.
Все~информационные операции должны выводить результат с~учётом всех имеющихся в~репозиториях пакетов,
а~не~ограничиваться множеством установленных.

\begin{description}

\itemВывод подробной информации о~пакете.[
Пользователь должен иметь возможность получить подробную информацию о~пакете,
включающую имя, версию (эпоху, версию и релиз), архитектуру, автора (\EN{packager}), \EN{URL} проекта,
лицензию, имя пакета с~исходными текстами, однострочное и развёрнутое описание.

\item[Поиск пакета.]
Пользователь должен иметь возможность произвести поиск пакета по~его имени (включая обработку регулярных выражений),
по~однострочному описанию и подробному описанию, указывая, какие из~перечисленных полей необходимо просматривать.

\item[Поиск пакетов, предоставляющих некоторый \EN{provides}.]
Пользователь должен иметь возможность запросить список всех пакетов,
содержащих указанный \EN{provides}.
При~запросе операции ограничения версии не~указывается,
но~система выводит список подходящих пакетов с~уточнением версии соответствующего \EN{provides}.

\item[Поиск нарушений целостности множества пакетов в~репозиториях.]
Пользователь должен иметь возможность произвести анализ содержимого подключенных репозиториев и получить список пакетов,
установка которых невозможна по~причине отсутствия пакетов, удовлетворяющих некоторые зависимости.
Список должен содержать имя пакета и соответствующую зависимость, порождающую ошибку.

\item[Вывод бинарных пакетов, получаемых сборкой указанного пакета с~исходными текстами.]
Пользователь должен иметь возможность получить список пакетов,
которые были созданы путём сборки указанного пакета с~исходными текстами.
Требуется возможность указания несколько пакетов с~исходными текстами одновременно.

\item[Вывод информации о~зависимостях и конфликтов пакета.]
Пользователь должен иметь возможность просмотреть информацию 
о всех зависимостях и конфликтах указанного пакета, дополненную перечислением пакетов, удовлетворяющих каждому пункту выводимых данных.
Требуется явно помечать зависимости, разрешение которых невозможно удовлетворить,.

\item[Вывод зависимых пакетов.]
Пользователь должен иметь возможность просмотреть список пакетов, зависимых от~указанного.
Другими словами, требуется вывести список пакетов, к~удалению которых приведёт удаление указанного.
Необходимо иметь возможность просмотреть как список пакетов, зависимых напрямую, 
так и список косвенно зависимых пакетов, включая транзитивные замыкания.
Анализироваться должны только пакеты, установленные в~систему.

\item[Вывод информации о~зависимостях между репозиториями.]
Пользователь должен иметь возможность просмотреть, 
является~ли содержимое некоторого репозитория целостным (каждый пакет может быть установлен в~систему без~подключения сторонних репозиториев)
или для~работы должны задействоваться пакеты из~других репозиториев. 

\end{description}

\section{Структура пакета}
\label{pkgstruct}

\section{Репозитории пакетов}
\section{Конфигурационные возможности}

